\begin{appendices}

\chapter{Vulnerable C Program}
\label{app:vuln-full}

The vulnerable test program fuzzed using \texttt{afl}. Program can be
downloaded at: \url{https://raw.githubusercontent.com/nskinkel/417paper/master/files/msg-parse.c}

Unsafe code begins in Line 126.

\renewcommand\mylstcaption{Caption goes here.}
\begin{TCBlisting}[language={[ANSI]C},basicstyle=\scriptsize,caption={\mylstcaption}]
#include <stdio.h>
#include <stdlib.h>
#include <stdint.h>
#include <string.h>

/**
 * A simple test program that parses a msg_t from an input file and 
 * prints the username and address. The parse_msg() function is
 * intentionally written with three unsafe statements:
 *
 *     - if the input file is well-formed (the user tells the truth
 *       about the lengths of their address and username), the program
 *       proceeds as expected without any memory violations
 *
 *     - however, if the length fields of the input file are
 *       inaccurate, the program will attempt may attempt to either
 *       read from or write to unowned memory, depending on the input
 *       file fields
 *
 * See the UNSAFE statements in the parse_msg() function for details.
 *
 * This program is interesting because, with good input, memory
 * analysis tools like valgrind won't complain.
 */
#define MIN_MSG_SIZE 1 + 1 + 1 + 2 + \
                     sizeof(char*) + sizeof(char*)
#define MSG_TYPE_CLIENT_HELLO 1

typedef struct {
    uint8_t type;       // What kind of message is this?
    uint8_t addr_len;   // How long, in BE bytes, is addr field.
    uint8_t uname_len;  // How long, in bytes, is uname field.
    unsigned short len; // Total length, in bytes, of this message.
    char *addr;         // Email address.
    char *uname;        // Username.
} msg_t;

/**
 * Exit nicely and print a message on stderr if we fail.
 */
void
errexit(char *reason) {
    fprintf(stderr, reason);
    exit(1);
}

/**
 * Read a file into memory. Allocates enough memory in buf to hold the
 * file contents and returns size of buf.
 *
 * Fail gracefully and immediately if any system calls return an error
 */
size_t
read_file(const char *const fname, uint8_t **buf) {
   	FILE *file;
	size_t len;

    if (!(file = fopen(fname, "r"))) {
        errexit("fopen() failed.\n");
    }
    if (fseek(file, 0, SEEK_END)) {
        errexit("fprintf() failed.\n");
    }
    if ((len = ftell(file)) == -1) {
        errexit("ftell() failed.\n");
    }

    rewind(file);

    if ((*buf = malloc(len))) {
        if (fread(*buf, 1, len, file) != len) {
            errexit("fread() failed.\n");
        }
    } else {
        errexit("malloc() failed.\n");
    }
    if (fclose(file)) {
        errexit("fclose() failed.\n");
    }

    return len;
}

/**
 * Parse a msg_t struct from buf.
 *
 * This function is intentionally unsafe, see UNSAFE blocks for
 * more details.
 */
msg_t *
parse_msg(uint8_t *buf, size_t buf_len) {
    size_t idx = 0;
    msg_t *msg;

    if (!(msg = malloc(sizeof(msg_t)))) {
        errexit("malloc() for msg failed.\n");
    }

    // Fail gracefully if file is too short to possibly contain
    // a valid msg.
    if (buf_len < MIN_MSG_SIZE) {
        errexit("Impossibly short input buffer.\n");
    }

    if (buf[idx] != MSG_TYPE_CLIENT_HELLO) {
        errexit("Unrecognized message type.\n");
    }
    // read msg type
    msg->type = buf[idx++];
    // read addr length
    msg->addr_len = buf[idx++];
    // read uname length
    msg->uname_len = buf[idx++];
    // read overall msg len
    msg->len = (buf[idx] << 8) | buf[idx+1];
    idx += 2;

    if (!(msg->addr = malloc(msg->addr_len+1))) {
        errexit("malloc() failed for msg->addr.\n");
    }

    if (!(msg->uname = malloc(msg->uname_len+1))) {
        errexit("malloc() failed for msg->uname.\n");
    }

    // UNSAFE 1: read of buf[idx] without checking
    //           idx+msg->addr_len < buf_len
    memcpy(msg->addr, &(buf[idx]), msg->addr_len);
    idx += msg->addr_len;
    msg->addr[msg->addr_len] = '\0';

    // UNSAFE 2: read of buf[idx] without checking
    //           idx+msg->uname_len < buf_len
    memcpy(msg->uname, &(buf[idx]), msg->uname_len);
    msg->uname[msg->uname_len] = '\0';

    // for user privacy, zero-out message buffer when we're done
    // UNSAFE 3: writes to buf without checking
    //           msg->len <= buf_len
    memset(buf, 0, msg->len);

    return msg;
}

int
main(int argc, char *argv[]) {
    uint8_t *data;
    size_t len;
    
    if (argc != 2) {
        errexit("Usage: msg-parse msg-file\n");
    }

    len = read_file(argv[1], &data);
    msg_t *msg = parse_msg(data, len);    

    printf("Username: %s\n", msg->uname);
    printf("Contact: %s\n", msg->addr);

    free(msg->uname);
    free(msg->addr);
    free(msg);
    free(data);

    return 0;
}
\end{TCBlisting}

\chapter{Seed Input Tool}
\label{app:vuln-seed}

A small Python program to generate a set of \texttt{afl} input seed files
for the message format used in the program listed in Appendix \ref{app:vuln-full}.
This tool can be downloaded at: \url{https://raw.githubusercontent.com/nskinkel/417paper/master/files/gen-seeds.py}

\renewcommand\mylstcaption{Caption goes here.}
\begin{TCBlisting}[language=Python,basicstyle=\scriptsize,caption={\mylstcaption}]
#! /usr/bin/env python

import random
import string
import struct


FNAME_PREFIX    = 'afl-seed-msg'
MAX_UNAME_LEN   = 64
MAX_ADDR_LEN    = 64
OUTPUT_NUM      = 10


def get_one():
    return random.choice(string.letters+string.digits)


def get_range(upper):
    l = random.randrange(1, upper)
    return [get_one() for c in xrange(0, l)]    
    

for n in xrange(0, OUTPUT_NUM):
    addr = ''.join(get_range(MAX_ADDR_LEN))
    uname = ''.join(get_range(MAX_UNAME_LEN))

    with open(FNAME_PREFIX+"-"+str(n), 'w') as f:
        f.write(struct.pack(
            '!BBBH{0}s{1}s'.format(len(addr), len(uname)),
            1, len(addr), len(uname),
            1+1+1+2+len(addr)+len(uname), addr, uname)
        )
\end{TCBlisting}

\chapter{\texttt{afl}-Generated Test Cases}
\label{app:test-cases}

The following figures show \texttt{afl}-generated test cases that trigger each
of the memory safety vulnerabilities in the program shown in Appendix
\ref{app:vuln-full}. Figure \ref{fig:crash1} shows a test
case that triggers the vulnerability on Line 128, Figure
\ref{fig:crash2} shows a test case that triggers the vulnerability on Line 134,
and Figure \ref{fig:crash3} shows a test case that triggers the vulnerability
on Line 140.

Full test case files can be downloaded at: \url{https://github.com/nskinkel/417paper/tree/master/files/afl-test-cases}

\begin{figure}
	\begin{lstlisting}[language={},basicstyle=\small]
     01 9c 02 00 23 6e 57 47  59 50 51 49 53 63 57 47
     6c 49 35 71 6b 42 59 4b  4b 36 31 76 58 56 68 73
     6f 57 68\end{lstlisting}
\caption{\texttt{afl}-generated file that triggers \texttt{UNSAFE 1}.}
\label{fig:crash1}
\end{figure}

\begin{figure}
	\begin{lstlisting}[language={},basicstyle=\small]
	 01 1e 02 00 23 6e 57 47  59 50 51 49 53 63 57 47
	 6c 49 35 71 6b 42 59 4b  4b 36 31 76 58 56 68 73
	 6f 57 68\end{lstlisting}
\caption{\texttt{afl}-generated file that triggers \texttt{UNSAFE 2}.}
\label{fig:crash2}
\end{figure}

\begin{figure}
	\begin{lstlisting}[language={},basicstyle=\small]
	 01 02 17 80 1e 56 67 4c  34 78 76 59 61 72 71 7a
	 34 75 34 77 54 31 4f 74  69 48 46 61 37 75\end{lstlisting}
\caption{\texttt{afl}-generated file that triggers \texttt{UNSAFE 3}.}
\label{fig:crash3}
\end{figure}

\end{appendices}
