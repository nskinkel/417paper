\chapter{The \texttt{afl} Approach}

% TODO: mention somewhere afl is targetted mainly for binary inputs

At a high level, \texttt{afl} has only a two goals: given some input corpus,
as quickly and reliably as possible:

\begin{enumerate}
    \item Create test cases that reach new program states
    \item Identify inputs that cause the process to die due to some
    OS signal (e.g. \texttt{SIGSEV} or \texttt{SIGABRT})
\end{enumerate}

That's it. \texttt{afl} just repeatedly mutates input files in order to
``evolve'' more interesting test cases out of thin air.
\texttt{afl}'s novelty compared to other fuzzing tools comes from its
efficiency and core feedback loop: special instructions are inserted into the
target binary at compile-time, and during program execution every branch
statement reports back to the controlling process whenever it is executed.
Thus, \texttt{afl} can efficiently classify newly-mutated test cases based on
the number and diversity of branch statements they cause to be executed.

Test cases that cause a program to crash or hang are archived along with
execution metadata.

\section{Core Operations}

% TODO
% mention some stuff about the overall layout so the rest of this makes
% sense. in particular, there is a controlling process that spins up
% other child processes.

% mention that program gets run with a special wrapper, afl-fuzz

% in-depth overview

\subsection{Compile-time Instrumentation}

\texttt{afl} injects code at \textit{every} branch point in the target
binary and captures both overall branch coverage and a rough measure of branch 
hit counts. Figure \ref{fig:injection} shows the simplified version of the code
injected around branch statements\cite{afl-whitepaper}.

\begin{figure}[H]
    \begin{lstlisting}[language={[ANSI]C}]
cur_location = <COMPILE-TIME RANDOM>;
shared_mem[cur_location ^ prev_location]++;
prev_location = cur_location >> 1;
\end{lstlisting}
\caption{Simplified version of the code injected around branch statements}
\label{fig:injection}
\end{figure}

% what we do with branch statements
% example?


\subsection{Classifying Program State}

\texttt{afl} classifies program state by labeling branch statements
as a $(src, dst)$ 2-tuple.

\texttt{afl} maintains a global array 


% (src, dst) tuples in global shared map
% what makes a new state

\subsection{Exploring New Program States}

% when branch is hit, update shared mem bucket
% when to save a test case

\section{Fuzzing Strategies}

\subsection{Test Case Evolution}

\subsection{Deterministic Fuzzing}

\subsection{Random Fuzzing}

\section{Optimizations}
% smart optimization means more states explored

\subsection{Pruning the Input Queue}

\subsection{Trimming Test Cases}

\subsection{Copy-on-write Process Clones}
