\chapter{Practical Evaluation}

This practical evaluation has 2 main parts:
\begin{enumerate}
    \item an \texttt{afl} test run on a sample program with some intentional
    memory safety vulnerabilities that can be triggered by specially-crafted
    input files
    \item a very brief overview of some real-world bugs and vulnerabilities
    found by \texttt{afl} as well as its overall effect on the security
    community
\end{enumerate}

\section{Fuzzing a Toy Program}

% TODO: set this up

\subsection{The Vulnerable Code}

Figure \ref{fig:vuln-snippet} shows the vulnerable code snippet from
the fuzzed program (for full source code, see Appendix \ref{app:vuln-full}).

\begin{figure}[H]
    \begin{lstlisting}[language={[ANSI]C}]
// UNSAFE: read of buf[idx] without checking
//         idx+msg->addr_len < buf_len
memcpy(msg->addr, &(buf[idx]), msg->addr_len);
idx += msg->addr_len;
msg->addr[msg->addr_len] = '\0';

// UNSAFE: read of buf[idx] without checking
//         idx+msg->uname_len < buf_len
memcpy(msg->uname, &(buf[idx]), msg->uname_len);
msg->uname[msg->uname_len] = '\0';                                         

// for user privacy, zero-out message buffer
// when we're done with it
// UNSAFE: writes to buf without checking
//         msg->len <= buf_len
memset(buf, 0, msg->len);
\end{lstlisting}
\caption{A vulnerable code snippet that, when given malicious or malformed data, will violate memory safety.}
\label{fig:vuln-snippet}
\end{figure}

\subsection{\texttt{afl} Results}

% TODO

\subsection{A Sample \texttt{afl} Test Case}

% TODO

\section{\texttt{afl} in the Real World}

% TODO

% table with truncated results

\section{Conclusion}

% TODO
