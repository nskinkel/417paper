\documentclass{beamer}

\mode<presentation> {

%\usetheme{default}
%\usetheme{AnnArbor}
%\usetheme{Antibes}
%\usetheme{Bergen}
%\usetheme{Berkeley}
%\usetheme{Berlin}
%\usetheme{Boadilla}
\usetheme{CambridgeUS}
%\usetheme{Copenhagen}
%\usetheme{Darmstadt}
%\usetheme{Dresden}
%\usetheme{Frankfurt}
%\usetheme{Goettingen}
%\usetheme{Hannover}
%\usetheme{Ilmenau}
%\usetheme{JuanLesPins}
%\usetheme{Luebeck}
%\usetheme{Madrid}
%\usetheme{Malmoe}
%\usetheme{Marburg}
%\usetheme{Montpellier}
%\usetheme{PaloAlto}
%\usetheme{Pittsburgh}
%\usetheme{Rochester}
%\usetheme{Singapore}
%\usetheme{Szeged}
%\usetheme{Warsaw}

%\usecolortheme{albatross}
\usecolortheme{beaver}
%\usecolortheme{beetle}
%\usecolortheme{crane}
%\usecolortheme{dolphin}
%\usecolortheme{dove}
%\usecolortheme{fly}
%\usecolortheme{lily}
%\usecolortheme{orchid}
%\usecolortheme{rose}
%\usecolortheme{seagull}
%\usecolortheme{seahorse}
%\usecolortheme{whale}
%\usecolortheme{wolverine}

%\setbeamertemplate{footline} % To remove the footer line in all slides uncomment this line
%\setbeamertemplate{footline}[page number] % To replace the footer line in all slides with a simple slide count uncomment this line
%\setbeamertemplate{navigation symbols}{} % To remove the navigation symbols from the bottom of all slides uncomment this line
}

\usepackage{graphicx}
\usepackage{multimedia}
\usepackage{booktabs}
\usepackage{float}
\usepackage{tikz}
\usepackage{scalefnt}
\usepackage{listings}
\usepackage{todonotes}
\usepackage{tabularx}
\usepackage{enumitem}
\usepackage{hyperref}
\usepackage{xcolor}
\usepackage{caption}


\setitemize{label=\usebeamerfont*{itemize item}
\usebeamercolor[fg]{itemize item}
\usebeamertemplate{itemize item}}

\definecolor{ao(english)}{rgb}{0.0, 0.5, 0.0}
\definecolor{aureolin}{rgb}{0.99, 0.93, 0.0}
\definecolor{bostonuniversityred}{rgb}{0.8, 0.0, 0.0}

\renewcommand{\tabularxcolumn}[1]{>{\small}m{#1}}
\renewcommand{\arraystretch}{2}

%% Listings definition for Go language
%% Go language reference : http://www.golang.org
%% Author : Uriel Corfa <uriel@corfa.fr>
%% Project home: https://bitbucket.org/korfuri/golang-latex-listings

\lstdefinelanguage{go}{
  % Keywords as defined in the BNF
  morekeywords=[1]{break,default,func,interface,%
    case,defer,go,map,struct,chan,else,goto,package,%
    switch,const,fallthrough,if,range,type,continue,%
    for,import,return,var,select},
  % Special identifiers, builtin functions
  morekeywords=[2]{make,new,nil,len,cap,copy,complex,%
    real,imag,panic,recover,print,println,iota,close,%
    closed,_,true,false,append,delete},
  % Basic types
  morekeywords=[3]{%
    string,int,uint,uintptr,double,float,byte,%
    int8,int16,int32,int64,int128,%
    uint8,uint16,uint32,uint64,uint128,%
    float32,float64,complex64,complex128,%
    rune},
  % Strings : "toto", 'toto', `toto`
  morestring=[b]{"},
  morestring=[b]{'},
  morestring=[b]{`},
  % Comments : /* comment */ and // comment
  comment=[l]{//},
  morecomment=[s]{/*}{*/},
  % Options
  sensitive=true
}

\lstset{
  language={go},
  basicstyle=\ttfamily\tiny
}

\title[afl]{\textbf{Modern Fuzzing with afl}}

\author{}
\institute[]
{
Anish Kunduru \\
Jon Osborne \\
Nik Kinkel \\
\medskip
}
\date{\today}

\begin{document}

\begin{titlepage}

\begin{center}

\Huge \textbf {Modern Fuzzing with \texttt{afl}}\\[0.5in]

\vspace{.1in}

\normalsize
\textbf{Anish Kunduru} \\
\textit{akunduru@iastate.edu} \\
\vspace{0.2cm}

\textbf{Jon Osborne} \\
\textit{osborj1@iastate.edu} \\
\vspace{0.2cm}

\textbf{Nik Kinkel} \\
\textit{nskinkel@iastate.edu} \\

\vfill

\textsc{Iowa State University}\\
\vspace{0.2cm}
Spring 2016

\end{center}

\end{titlepage}

% Slide 1
\begin{frame}
\frametitle{TODO}

% TODO

\end{frame}

% Slide 2
\begin{frame}
\frametitle{Earlier Approaches}
\begin{columns}
\column{0.4\textwidth}
\begin{itemize}
\item{Relatively simple, rely on injecting random data}
\item{Mutative Approach}
\item{Generative Approach}
\end{itemize}
\column{0.6\textwidth}
\begin{figure}
\frame{\includegraphics[width=.6\linewidth]{figures/fuzzingStringsMutative.png}}
\caption{Mutative fuzzing of a string.}
\end{figure}
\begin{figure}
\frame{\includegraphics[width=.6\linewidth]{figures/fuzzingAsAConceptGenerative.png}}
\caption{A generative fuzzer creates its own test cases.}
\end{figure}
\end{columns}
\end{frame}

% Slide 3
\begin{frame}
\frametitle{american fuzzy lop (\texttt{afl})}

\begin{columns}[c]

\column{0.5\textwidth}

\textbf{Goal}: find unique crashes.

\begin{itemize}
    \item Efficiency is coverage.
    \item Syntax is hard.
    \item State transitions matter.
    \item Automatic feedback loop is critical.
\end{itemize}

\end{columns}

\end{frame}

% Slide 4
\begin{frame}
\frametitle{\texttt{afl} and Program State}

\begin{columns}[c]

\column{0.5\textwidth}

\begin{itemize}
    \item \texttt{afl} state classification
    \item Tracking state on live executions
    \item \textit{New} states
    \item \textit{Unique} crashes
\end{itemize}

\end{columns}

\end{frame}

% Slide 5
\begin{frame}
\frametitle{Core Operation}

\begin{columns}[c]

\column{0.5\textwidth}

\begin{itemize}
    \item Pruned test case input queue
    \item Spawn new cow processes
    \item Save unique crashes and hangs
\end{itemize}

\column{0.5\textwidth}

TODO: some cool figure(s) here

\end{columns}
\end{frame}

% Slide 6
\begin{frame}
\frametitle{TODO}

% TODO

\end{frame}

% Slide 7
\begin{frame}
\frametitle{TODO}

% TODO

\end{frame}

% Slide 8
\begin{frame}[fragile]
\frametitle{Practical Evaluation}

\begin{columns}[c]

\column{0.4\textwidth}

\begin{figure}
\vspace*{-0.5cm}
\caption{Message structure}
\vspace*{-0.85cm}
\begin{lstlisting}[language=C,frame=single]
typedef struct {
    (*@\texttt{\color{blue}uint8\_t}@*)  type;
    (*@\texttt{\color{blue}uint8\_t}@*)  addr_len;
    (*@\texttt{\color{blue}uint8\_t}@*)  uname_len;
    (*@\texttt{\color{blue}uint16\_t}@*) len;
    (*@\texttt{\color{blue}char}@*)    *addr;
    (*@\texttt{\color{blue}char}@*)    *uname;
} msg_t;
\end{lstlisting}
\end{figure}

\begin{figure}
\vspace*{-1cm}
\caption{Vulnerable parsing code}
\vspace*{-1cm}
\begin{lstlisting}[language=C,frame=single]
// UNSAFE read
memcpy(msg->addr,
       &(buf[idx]),
       msg->addr_len);
idx += msg->addr_len;
msg->addr[msg->addr_len] = '\0';
// UNSAFE read
memcpy(msg->uname,
       &(buf[idx]),
       msg->uname_len);
msg->uname[msg->uname_len] = '\0';
// UNSAFE write
memset(buf, 0, msg->len);
\end{lstlisting}
\end{figure}


\column{0.6\textwidth}

\begin{center}
\begin{figure}
\vspace*{-0.6cm}
\caption{\texttt{afl} run}
\vspace*{-0.5cm}
\includegraphics[scale=0.32]{../figures/afl-run}
\end{figure}
\end{center}

\begin{center}
\begin{varwidth}{0.7\textwidth}
\begin{figure}
\vspace*{-1cm}
\begin{lstlisting}[numbers=none]
01 (*@\SoulColor\hl{9c}@*) 02 00 23 6e 57 47 59 50 51 49
53 63 57 47 6c 49 35 71 6b 42 59 4b
4b 36 31 76 58 56 68 73 6f 57 68\end{lstlisting}
\caption{\texttt{afl}-generated Test Case}
\end{figure}
\end{varwidth}
\end{center}


\end{columns}


\end{frame}

% Slide 9
\begin{frame}
\Huge{\centerline{Questions}}
\end{frame}


\end{document} 
